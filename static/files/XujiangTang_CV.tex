%-------------------------
% Academic CV for Xujiang Tang
% Based on standard LaTeX CV template
% Compile with: xelatex or pdflatex
%-------------------------

\documentclass[11pt,a4paper]{article}

% ===== Packages =====
\usepackage[margin=0.75in]{geometry}
\usepackage{titlesec}
\usepackage{enumitem}
\usepackage{hyperref}
\usepackage{fancyhdr}
\usepackage{xcolor}
\usepackage{fontawesome5}
\usepackage[utf8]{inputenc}
\usepackage[T1]{fontenc}
\usepackage{lmodern}

% ===== Colors =====
\definecolor{primary}{RGB}{0, 51, 102}
\definecolor{linkcolor}{RGB}{0, 102, 204}

% ===== Hyperlink Setup =====
\hypersetup{
    colorlinks=true,
    linkcolor=linkcolor,
    urlcolor=linkcolor,
    pdfauthor={Xujiang Tang},
    pdftitle={Xujiang Tang - Curriculum Vitae}
}

% ===== Section Formatting =====
\titleformat{\section}{\large\bfseries\color{primary}\scshape}{}{0em}{}[\titlerule]
\titlespacing*{\section}{0pt}{12pt}{6pt}

% ===== Header/Footer =====
\pagestyle{fancy}
\fancyhf{}
\renewcommand{\headrulewidth}{0pt}
\rfoot{\small\thepage}

% ===== Custom Commands =====
\newcommand{\cventry}[4]{
    \textbf{#1} \hfill #2 \\
    \textit{#3} \hfill \textit{#4} \\[2pt]
}

\newcommand{\cventryoneline}[2]{
    \textbf{#1} \hfill #2 \\[2pt]
}

% ===== Document =====
\begin{document}

% ===== Title =====
\begin{center}
    {\LARGE\bfseries\color{primary} Xujiang Tang} \\[6pt]
    {\large Curriculum Vitae} \\[8pt]
    \faEnvelope\ \href{mailto:txj_262538@163.com}{txj\_262538@163.com} \quad
    \faPhone\ +86 183-2782-9197 \\[3pt]
    \faGithub\ \href{https://github.com/TXJ2006}{github.com/TXJ2006} \quad
    \faOrcid\ \href{https://orcid.org/0009-0008-1127-6420}{0009-0008-1127-6420} \\[3pt]
    \faLinkedin\ \href{https://www.linkedin.com/in/\%E7\%85\%A6\%E6\%B1\%9F-\%E5\%94\%90-7a1513392}{LinkedIn} \quad
    \faGlobe\ \href{https://txj2006.github.io}{txj2006.github.io}
\end{center}

% ===== Education =====
\section{Education}

\cventry{Yangtze University College of Arts and Sciences}{Jingzhou, Hubei, China}
{B.Sc. in Mathematics and Applied Mathematics}{Sep. 2023 -- Jun. 2027 (Expected)}
\textbf{GPA: 3.5 / 4.0}

\vspace{4pt}
\textbf{Key High-Achievement Courses:}
\begin{itemize}[nosep, leftmargin=1.5em]
    \item Optimization Theory --- \textbf{97/100} (Ranked 1st in the Major)
    \item Numerical Computation Methods --- \textbf{96/100}
    \item Mathematical Modeling, Data Structures, Ordinary Differential Equations, Probability \& Mathematical Statistics
\end{itemize}

% ===== Publications =====
\section{Submitted / Under-Review Papers \& Manuscripts}

{\small\textit{Focus: Learning Theory, Optimization Algorithms, Stochastic Processes and Causal Inference}}

\begin{enumerate}[leftmargin=1.5em, itemsep=6pt]
    \item \textbf{X. Tang} and C. Fan (Corresponding Author). \textit{Out-of-Distribution Generalization Error Bounds for Successful Prediction of Deep Autoregressive Algorithms.} \\
    Submitted to \textbf{ICML 2026} $|$ Under Review

    \item \textbf{X. Tang}, et al. \textit{Geometric Disentanglement of Causal Cycles and Latent Confounders via Non-Gaussian Heterogeneity.} \\
    Submitted to \textbf{UAI 2026} (Causal Inference \& Statistical Learning) $|$ Under Review

    \item \textbf{X. Tang}. \textit{Quartic Difficulty: Assessing the Learnability of Unsupervised Learning Algorithms.} \\
    Submitted to \textbf{COLT} (Generalization Bounds and Theoretical Guarantees)

    \item \textbf{X. Tang}. \textit{Robust Optimal Reinsurance and Investment with Inflation Risk: A Game-Theoretic Approach and Explicit Solutions.} \\
    Submitted to \textbf{AIMS Mathematics}

    \item \textbf{X. Tang} and D. Hu. \textit{Geometric Manifold Rectification: Inducing Neural Collapse for Robust Representation Alignment.} \\
    Submitted to \textbf{ICLR 2026 Workshop}

    \item \textbf{X. Tang}, et al. \textit{Theoretical Analysis of Transformer Architecture.} \\
    Submitted to an \textbf{EI-indexed International Conference} (MIT Remote Research Group)
\end{enumerate}

% ===== Research Experience =====
\section{Research Experience}

\cventry{Independent Research on Dynamical Systems and Machine Learning}{Remote / On-Campus}
{Project Lead}{Mar. 2024 -- Present}
\begin{itemize}[nosep, leftmargin=1.5em]
    \item \textbf{Koopman Operator Theory:} Collaborated with the research group of Prof. Yanbing Jia (Henan University of Science and Technology) to solve long-sequence modeling problems.
    \item Proposed a Koopman Operator Linearization Skeleton system for nonlinear dynamical modeling.
    \item Validated the system on multiple chaotic datasets, outperforming baseline deep learning models.
\end{itemize}

\vspace{6pt}

\cventry{Stochastic Modeling and Quantitative Finance}{Independent Team Project}
{Team Lead}{Oct. 2023 -- Apr. 2024}
\begin{itemize}[nosep, leftmargin=1.5em]
    \item \textbf{Rough Volatility \& MCMC:} Led a team to investigate the rough volatility characteristics of financial markets.
    \item Overcame computational bottlenecks of MCMC methods in stochastic process generation.
    \item Designed a novel algorithm to resolve inductive bias in matrix representation; optimized parameter estimation strategies.
    \item Open-sourced the model framework on GitHub; manuscript in preparation for a top-tier journal.
\end{itemize}

\vspace{6pt}

\cventry{Biomedical Data Science (Spatial Transcriptomics)}{Northeastern University}
{Research Assistant $|$ Advisor: Dr. Dayu Hu}{Sep. 2024 -- Jan. 2025}
\begin{itemize}[nosep, leftmargin=1.5em]
    \item Assisted in designing single-cell data clustering algorithms, integrating a geometric perspective into computational pipelines.
    \item Optimized computation and processing workflows for high-dimensional biological data.
\end{itemize}

% ===== Internships =====
\section{Internship \& Professional Experience}

\cventry{Peking University Open-Source Large Model Group}{Remote}
{Research Engineering Intern}{Jun. 2024 -- Aug. 2024}
\begin{itemize}[nosep, leftmargin=1.5em]
    \item Participated in engineering deployment of Large Language Models (LLMs) and Retrieval-Augmented Generation (RAG) tasks, improving the model's external knowledge retrieval performance.
\end{itemize}

\vspace{6pt}

\cventry{Deloitte \& Guolian Securities}{China}
{Data Analysis Intern}{Jul. 2024 -- Sep. 2024}
\begin{itemize}[nosep, leftmargin=1.5em]
    \item Applied statistical methods to securities research and developed quantitative workflows for financial reporting.
    \item Led team projects to solve business problems via data-driven approaches.
\end{itemize}

% ===== Honors =====
\section{Honors \& Awards}

\begin{itemize}[leftmargin=1.5em, itemsep=3pt]
    \item \textbf{Silver Medal}, Kaggle Jigsaw Toxic Comment Classification Challenge
    \item \textbf{Regional Gold Medal}, WorldQuant International Quant Championship
    \item \textbf{Third Prize}, Graduate AI Forum, University of Chinese Academy of Sciences (UCAS) --- \textit{Only undergraduate recipient}
    \item \textbf{Multiple National Second Prizes} in mathematical modeling competitions (Future Cup, Central China Cup, Higher Education Press Cup); Led university mathematical modeling team formation and training
    \item \textbf{Reviewer}, ICLR 2026 Workshop (GRAm)
    \item \textbf{Participation Award}, NeurIPS 2025 Open Problems in WiML \& OPP Competitions
\end{itemize}

% ===== Skills =====
\section{Skills}

\begin{itemize}[leftmargin=1.5em, itemsep=3pt]
    \item \textbf{Mathematics:} Convex Optimization, Numerical Analysis, Stochastic Processes, Differential Equations, Causal Inference
    \item \textbf{Programming:} Python (PyTorch, NumPy, Pandas), MATLAB, R, C++
    \item \textbf{Tools:} \LaTeX, Git/GitHub, Linux
    \item \textbf{Languages:} English (Professional Fluency), Mandarin (Native)
\end{itemize}

\vfill
\begin{center}
    \small\textit{Last updated: February 2026}
\end{center}

\end{document}
